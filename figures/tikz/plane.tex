\documentclass{standalone}
\usepackage{tikz} 
\usetikzlibrary{shapes}
\usepackage{pgfplots}
\usepgfplotslibrary{fillbetween}
\usepackage[active,tightpage]{preview}  %generates a tightly fitting border around the work
\PreviewEnvironment{tikzpicture}
\setlength\PreviewBorder{2mm}
\usepackage{xcolor}
\definecolor{myred}{RGB}{196,19,47} 
\definecolor{myblue}{RGB}{0,139,139}

\begin{document}
		
\begin{tikzpicture}[scale=5]
	
%erstes Bild	
	
	%Füllen der Grundfläche
	\path[name path=Grundfläche,fill,myblue!30] (0,0,0) -- (0,0,-1) -- (1.2,0,-1) -- (1.2,0,0) -- (0,0,0);
	
	%Rand
	\draw[thin] (0,0,0) -- (1.2,0,0);
	\draw[thin] (0,0,0) -- (0,0,-1);
	\draw[thin] (0,0,-1) -- (1.2,0,-1);
	\draw[thin] (1.2,0,-1) -- (1.2,0,0);
	
	%Vektoren
	\draw[ultra thick,color=myred,->] (0.4,0,-0.4) -- (0.4,0,-0.8);
	\draw[ultra thick,color=myred,->] (0.8,0,-0.4) -- (0.8,0,-0.8);
	\node at (0.03,0,-1.25) {\Large$\vec{v}$};
	\draw[->] (0.07,0,-1.12) [out=270, in=150] to  (0.36,0,-0.6);
	
	%Punkte auf Karte
	\filldraw (0.4,0,-0.4)circle (0.2pt);
	\node at (0.45,0,-0.2) {\Large$P$};
	\filldraw (0.8,0,-0.4)circle (0.2pt);
	\node at (0.85,0,-0.2) {\Large$P'$};
	

	
	%gestrichelte Verbindung
	\draw[dashed,thick,-] (0.78,0,-0.4) -- (0.42,0,-0.4); 
		
	\end{tikzpicture}
\end{document}